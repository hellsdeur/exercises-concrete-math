\section*{Lista 1}

\begin{enumerate}
	\item \textbf{Escolha uma linguagem de programação, implemente as funções de
	recorrência e exiba os seis primeiros termos de cada sequência. Inclua o código fonte das funções na resposta.}
	\begin{enumerate}
		\item $a_1 = 5$ e $a_n = a_{n-1}+3, \forall n > 1$
		\lstinputlisting{../src/1_1a.cpp}
		\pagebreak
		\item $b_1 = 2$ e $b_n = b_{n-1}^2, \forall n > 1$
		\lstinputlisting{../src/1_1b.cpp}
		\item $c_1 = 0$ e $c_n = 2c_{n-1}+n, \forall n > 1$
		\lstinputlisting{../src/1_1c.cpp}
	\end{enumerate}

	\item \textbf{Escolha uma linguagem de programação e escreva um programa para
	receber uma sequência numérica e informar se a sequência é um P.A ou não. Caso seja uma P.A, o programa deve informar se a P.A é crescente, constante ou decrescente.}
	\lstinputlisting{../src/1_2.cpp}
	
	\item \textbf{Sabendo que o primeiro termo é igual a 3 e a razão é igual a 5, calcule o 17 o termo de uma P.A.}
	$$ a_n = a_1 + (n-1)^n $$
	$$ a_{17} = 3 + (17-1)5 $$
	$$ a_{17} = 3 + 16 \cdot 5 $$
	$$ a_{17} = 83 $$
	
	\item \textbf{Sabendo que o primeiro termo é igual a -8 e o vigésimo igual a 30, calcule a razão da P.A.}
	$$ a_n = a_1 + (n-1)r $$
	$$ 30 = -8 + (20-1)r $$
	$$ 30 = -8 + 19r $$
	$$ r = \frac{30+8}{19} $$
	$$ r = \frac{38}{19} $$
	$$ r = 2 $$
	
	\item \textbf{Escolha uma linguagem de programação e escreva um programa para receber os extremos de uma P.A, o valor de K e calcule a interpolação dessa P.A.}
	\lstinputlisting{../src/1_5.cpp}
	
	\item \textbf{Calcule a P.A em que a soma dos n primeiros termos é igual a $n^2 + 2n$.}
	$$ S_n = n^2 + 2n$$
	$$ S_1 = 1^2 + 2 \cdot 1 = 3 \quad \therefore \quad a_1 = 3 $$
	$$ S_2 = 2^2 + 2 \cdot 2 = 8 \quad \therefore \quad a_2 = S_2 - S_1 = 8 - 3 = 5 $$
	$$ S_3 = 3^2 + 2 \cdot 3 = 15 \quad \therefore \quad a_3 = S_3 - S_2 = 15 - 8 = 7 $$
	$$ S_4 = 4^2 + 2 \cdot 4 = 24 \quad \therefore \quad a_4 = S_4 - S_3 = 24 - 15 = 9 $$
	$$ S_5 = 5^2 + 2 \cdot 5 = 35 \quad \therefore \quad a_5 = S_5 - S_4 = 35 - 22 = 11 $$
	
	\item \textbf{Escolha uma linguagem de programação e escreva um programa para receber uma sequência numérica e informar se a sequência é um P.G ou não. Caso seja uma P.G, o programa deve informar se a P.G é crescente, constante, decrescente, alternante ou estacionária.}
	\lstinputlisting{../src/1_7.cpp}
	
	\item \textbf{Escolha uma linguagem de programação e escreva um programa para receber os extremos de uma P.G, o valor de K e calcule a interpolação dessa P.G.}
	\lstinputlisting{../src/1_8.cpp}
	
	\item \textbf{Escolha uma linguagem de programação e escreva um programa para receber uma sequência numérica. Se a sequência numérica for uma P.G, informe a produto e a soma dos termos dessa P.G. Caso contrário, informe que a sequência não é uma P.G.}
	\lstinputlisting{../src/1_9.cpp}
	
	\item \textbf{Determine o valor de $n$ tal que $\sum_{i=3}^n 2^i = 4088$}
	
\end{enumerate}