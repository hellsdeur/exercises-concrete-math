\section*{Atividade aula 1}

\begin{enumerate}
	\item \textbf{Escolha uma linguagem de programação, implemente as funções de recorrência e exiba os seis primeiros termos de cada sequência. Inclua o código fonte das funções na resposta.}
	\begin{enumerate}
		\item $a_1 = 5$ e $a_n = a_{n-1}+3, \forall n > 1$
		\lstinputlisting{../src/1_1a.cpp}

		\item $b_1 = 2$ e $b_n = b_{n-1}^2, \forall n > 1$
		\lstinputlisting{../src/1_1b.cpp}

		\item $c_1 = 0$ e $c_n = 2c_{n-1}+n, \forall n > 1$
		\lstinputlisting{../src/1_1c.cpp}
	\end{enumerate}

	\item \textbf{Escolha uma linguagem de programação e escreva um programa para receber uma sequência numérica e informar se a sequência é um P.A ou não. Caso seja uma P.A, o programa deve informar se a P.A é crescente, constante ou decrescente.}
	\lstinputlisting{../src/1_2.cpp}
	\verb|[3:10]| - A função \verb|checar_pa| recebe um array contendo a sequência, a razão entre os dois primeiros elementos e o seu tamanho $n$. Se a diferença entre um termo da sequência e o seu anterior for diferente da razão, a sequência não é P.A. Caso a razão se mantenha constante pra cada par de elementos, a sequência é P.A. \\
	\verb|[12:29]| - \verb|categorizar_pa| recebe apenas a sequência e seu tamanho. Ao calcular a razão entre o segundo e primeiro termo, é chamada a função \verb|checar_pa|, que dependendo do seu retorno e do valor da razão, classifica a sequência em "não P.A.", P.A. crescente, P.A. decrescente ou P.A. constante.
	
	\item \textbf{Sabendo que o primeiro termo é igual a 3 e a razão é igual a 5, calcule o 17 o termo de uma P.A.}\\
	O enésimo termo de uma P.A., onde o primeiro termo é $a_1$ e a razão é $r$, é dado por:
	$$ a_n = a_1 + (n-1)r $$
	Tomando $a_1 = 3$, $r=5$ e $n=17$, temos:
	$$ a_{17} = 3 + (17-1)5 $$
	$$ a_{17} = 3 + 16 \cdot 5 $$
	$$ a_{17} = 3 + 80 $$
	$$ a_{17} = 83 $$
	
	\item \textbf{Sabendo que o primeiro termo é igual a -8 e o vigésimo igual a 30, calcule a razão da P.A.}\\
	Partindo da definição do enésimo termo de uma P.A., a razão pode ser encontrada por:
	$$ a_n = a_1 + (n-1)r $$
	$$ (n-1)r = a_n - a_1 $$
	$$ r = \frac{a_n - a_1}{n-1} $$
	De posse de $a_1 = -8$ e $a_n = a_{20} = 30$, a razão desta P.A. é:
	$$ r = \frac{30 - (-8)}{20-1} $$
	$$ r = \frac{38}{19} $$
	$$ r = 2 $$
	
	\pagebreak
	\item \textbf{Escolha uma linguagem de programação e escreva um programa para receber os extremos de uma P.A, o valor de K e calcule a interpolação dessa P.A.}
	\lstinputlisting{../src/1_5.cpp}
	\verb|[3:10]| - O procedimento \verb|interpor_pa| recebe uma sequência vazia, os extremos da P.A. e a quantidade $k$ de elementos entre os extremos. De posse da razão, cada elemento da P.A. é inserido na sequência, começando do primeiro até o último, conforme o valor da razão.\\
	\verb|[23:33]| - Como o tamanho da sequência ($k+2$) depende da entrada e não é mais alterado, optou-se pela alocação dinâmica do array, ao invés de estruturas mais sofisticadas.
		
	\item \textbf{Calcule a P.A em que a soma dos n primeiros termos é igual a $n^2 + 2n$.}\\
	Conhecendo a expressão da soma dos $n$ termos, podemos encontrar a soma $S_1$ do primeiro termo, que nada mais é do que o primeiro termo:
	$$ S_n = n^2 + 2n$$
	$$ S_1 = 1^2 + 2 \cdot 1 = 3 \quad \therefore \quad a_1 = 3 $$
	Encontrando a soma $S_2$ dos dois primeiros termos, podemos encontrar o segundo termo, pois ele é a diferença entre $S_2$ e $S_1$. Para os demais termos, o procedimento é semelhante:
	$$ S_2 = 2^2 + 2 \cdot 2 = 8 \quad \therefore \quad a_2 = S_2 - S_1 = 8 - 3 = 5 $$
	$$ S_3 = 3^2 + 2 \cdot 3 = 15 \quad \therefore \quad a_3 = S_3 - S_2 = 15 - 8 = 7 $$
	$$ S_4 = 4^2 + 2 \cdot 4 = 24 \quad \therefore \quad a_4 = S_4 - S_3 = 24 - 15 = 9 $$
	$$ S_5 = 5^2 + 2 \cdot 5 = 35 \quad \therefore \quad a_5 = S_5 - S_4 = 35 - 22 = 11 $$
	Pode-se verificar a formação de uma progressão aritmética de razão $2$, portanto os demais elementos sempre aumentarão com esta razão.\\
	Generalizando, $a_n = S_n - S_{n-1} = (n^2 + 2n) - ((n-1)^2 + 2(n-1))$, para $S_1 = a_1 = 3$.
	
	\item \textbf{Escolha uma linguagem de programação e escreva um programa para receber uma sequência numérica e informar se a sequência é um P.G ou não. Caso seja uma P.G, o programa deve informar se a P.G é crescente, constante, decrescente, alternante ou estacionária.}
	\lstinputlisting{../src/1_7.cpp}
	\verb|[6:8]| - Excepcionalmente nessa solução é possível fazer a entrada de sequência de números reais fracionários, e.g. $1/3$. Para solucionar o problema de precisão na representação de pontos flutuantes foi criada a função \verb|chao| que arredonda um número real para baixo, com precisão de 6 casas decimais.\\
	\verb|[10:35]| - Para a entrada de números reais fracionários, foi criada a função \verb|parse_stof| que converte strings em valores numéricos, além de verificar se há sinal de divisão \verb|'/'| dentre os números da sequência inserida. Caso exista, é efetuada a divisão entre o numerador e o denominador. Desse modo, é possível representar com mais exatidão dízimas periódicas.\\
	\verb|[37:46]| - Não há função nativa em C++ para dividir uma string por caractere, como em outras linguagens. A função \verb|split| foi criada com esse intuito, dividindo a stream de strings inserida pelo usuário em strings divididas por espaços simples. Em seguida a função \verb|parse_stof| é chamada para a conversão das strings em floats.\\
	\verb|[48:55]| - É retornado de \verb|checar_pg| um booleano indicando se a sequência informada é uma P.G., baseada na razão $q$ entre os dois primeiros números.\\
	\verb|[57:81]| - A P.G. é classificada de acordo com o valor de $q$, e, quando necessário, do sinal dos termos da sequência. Caso $q > 1$ para termos positivos ou $0 < q < 1$ para termos negativos, a P.G. é crescente. Caso $0 < q < 1$ para termos negativos ou $q > 1$ para termos positivos, a P.G. é decrescente. Se $q = 1$ para termos não nulos, a P.G. é constante. Se $q < 0$, os termos alternam o sinal. Para $q = 0$, a P.G. é estacionária.
	
	\item \textbf{Escolha uma linguagem de programação e escreva um programa para receber os extremos de uma P.G, o valor de K e calcule a interpolação dessa P.G.}
	\lstinputlisting{../src/1_8.cpp}
	\verb|[4:12]| - \verb|interpor_pa| recebe a sequência vazia, o extremo inferior $a_1$, o extremo superior $a_ n$ e o valor de $k$ termos intermediários. A razão entre elementos consecutivos $q = \sqrt[k+1]{\frac{a_n}{a_1}}$ é usada para incrementar geometricamente o valor de $a_1$ a cada iteração no array que armazena a sequência.
	
	\item \textbf{Escolha uma linguagem de programação e escreva um programa para receber uma sequência numérica. Se a sequência numérica for uma P.G, informe a produto e a soma dos termos dessa P.G. Caso contrário, informe que a sequência não é uma P.G.}
	\lstinputlisting{../src/1_9.cpp}
	\verb|[5:7]| - Desta vez não se fará o uso de entradas em forma de fração, mas manteve-se a função de arrendondamento para baixo, pois os resultados das divisões a seguir podem ser fracionários e não serem representados de forma precisa, ocasionando inconsistência ao compará-los com outros números fracionários.\\
	\verb|[9:16]| - \verb|checar_pg| verifica se a razão entre cada elemento da sequência e seu antecessor é igual à razão entre os dois primeiros.
	\verb|[18:33]| - O procedimento \verb|soma_produto_pg| recebe a sequência e seu tamanho e verifica se é uma P.G. Se for, ocorre a aplicação da soma da P.G, $S_{P.G.} = \frac{a_1q^n-a_ 1}{q-1}$, e do produto da P.G., $P_{P.G.} = a_1^n\cdot q^{\frac{n(n-1)}{2}}$. Em seguida, os resultados são apresentados.

	\item \textbf{Determine o valor de $n$ tal que $\sum_{i=3}^n 2^i = 4088$}\\
	Para encontrar a soma de $i=3$ até $n$, basta notar que a soma de $i=1$ até $n$ pode ser representada da seguinte forma:
	$$\sum_{i=1}^{n} 2^i = \sum_{i=1}^{2} 2^i + \sum_{i=3}^{n} 2^i$$
	O lado esquerdo representa a soma de uma progressão geométrica.
	$$\frac{a_1q^n-a_1}{q-1} = \sum_{i=1}^{2} 2^i + \sum_{i=3}^{n} 2^i$$
	Tomando $a_1 = 2^1 = 2$ e $q = 2$, e substituindo os valores que já conhecemos do lado direito obtemos:
	$$\frac{2\cdot 2^n-2}{2-1} = (2+4) + 4088$$
	$$2\cdot 2^n-2 = 6 + 4088$$
	$$2\cdot2^n = 4094+2$$
	$$2^n = \frac{4096}{2}$$
	$$2^n = 2048$$
	$$2^n = 2^{11}$$
	$$n = 11$$
	
\end{enumerate}